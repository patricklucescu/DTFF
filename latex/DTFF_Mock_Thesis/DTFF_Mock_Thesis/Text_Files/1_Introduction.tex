\section{Introduction}

Financial times series exhibit sudden breaks which lead to processes with completely other characteristics. Early identification of such regime switches (RS) are of major importance for strategic asset allocation (SAA). 

We propose a completely model independent approach for SAA in the context of RS, thereby allowing the data to speak for itself. In a first step, using a broad pool of different asset classes, we focus on a  hierarchical clustering-based asset allocation (HCBAA) which ensures us an 'all-weather' portfolio based on historical correlations. In a second step, we use the long lasting predictive power of news sentiment to decide if we are currently entering a bull or bear market cycle. Based on this insight we decide then if we should overweight bonds or equities for the next holding period. 

Our work is an extension of the research done by \citet{enhPortOpti} who focused on enhancing the traditional \citet{black1992global} SAA models with a behavioral based approach based on news sentiment. We circumvent the usage of a model by relying on the HCBAA approach which was first introduced by \citet{raffinot2017hierarchical} thereby extending the initial work of \citet{de2016building}. Applying classical and more novel hierarchical clustering methods to a selection of different asset classes, \citet{raffinot2017hierarchical} achieved truly diversified portfolios and achieved statistically better risk-adjusted performances. 

A short-coming of HCBAA, compared to the BL model, is the inability to introduce a forward looking estimate into the clustering procedure. However, we believe that our crude model independent procedure of over weighting specific asset classes depending on the differentiation between two cycles (bear or bull), is a more robust procedure. First, we are not required to create portfolios in order to induce our views into the weight generation procedure. Second, we circumvent all the common pitfalls of the classical mean-variance optimization procedure. 

Our results show that portfolios based on hierarchical clustering-based asset allocation methods are indeed efficient and are able to circumvent huge losses during times of high financial stress. Nevertheless, we are unable to improve our results by adjusting our risk exposure in times of high risks, which are predicted using macro news sentiment data.

The paper is structured as follows. In a first step, we position ourselves in the academic literature. We then discuss the methodology and the data with greater detail, followed by a presentation of the results. A final conclusion summarizes then main findings. 







%The academic literature has focused on identifying and predicting such changes in returns. Most of these regime switching (RS) models only allow for linear predictability, while others allow for non-linear regime dependent approaches (see for example \cite{ang2004regimes}).

