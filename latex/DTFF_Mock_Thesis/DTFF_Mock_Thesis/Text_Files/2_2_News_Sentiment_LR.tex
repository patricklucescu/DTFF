\subsection{News Sentiment}

In the past decade the literature on News Sentiment Analysis has advanced tremendously due to the technological improvements in evaluation of large data sets. This has been a direct result of the improvements of Natural Language Processing algorithms as well as the evolution of computational power of computers. On the other hand, news sentiment has become a core part of behavioural finance. Among the first papers that discuss this topic is \cite{tetlock2007giving}, who found a correlation between negative news and negative future equity returns. Since then the predictive power has been shown to be persistent for weeks and months (\citet{uhl2015s}) rather than days as it was shown by \citet{tetlock2007giving}.

While most of the academic papers on news sentiment focus on investment time horizons of days, weeks or months, \citet{enhPortOpti} attempt to use long-term news sentiment in an SAA framework. They use long-term news sentiment momentum, as constructed by \citeauthor{uhl2015s}, in a Black-Litterman (BL) framework together with the classical mean variance optimization in order to derive the optimal weights deviation from a predefined benchmark. \citeauthor{enhPortOpti} are thus able to enhance the Sharpe Ratios of the benchmark SAAs by almost $20\%$. Nevertheless, the use of BL and  mean variance optimization comes with its own drawbacks. Besides the dimensionality curse of both BL and mean variance, in the BL framework the portfolio manager must form linear views on the assets which are assumed to be uncorrelated. However, in practice views correlations might exist but can be hard to quantify. 







 