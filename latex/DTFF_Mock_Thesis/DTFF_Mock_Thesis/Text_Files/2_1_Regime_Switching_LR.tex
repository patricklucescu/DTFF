\subsection{Regime Switching}

There is clear evidence that not only expected returns but also volatility vary over time. Furthermore, it has been documented by \citet{longin1999correlation} that equity returns are highly correlated in times of high volatility, and that this relation is statistically significant. \citet{Ang2002InternationalAA} find that there are actually two regimes in the international equity market, and that these regimes exhibit the characteristics described by \citet{longin1999correlation}. As such, it is desirable from the strategic asset manager to adjust its exposure to asset classes according to the current market scenario. 

This has been coined in the literature as Regime Switching models. In the simplest form, regime switching (RS) models allows data to be drawn from two or more distributions which can be considered as our regimes. As such, at each point in time there is a probability that the process will be drawn from the same distribution or it will switch to another one. Of major importance is the work of \citet{hamilton1989new} which uses a Markov process to model the interplay of the different regimes.

\citet{ang2004regimes} further develop regime switching strategies and show that a global manager can add value by considering two scenarios, a bear or a bull market. However, it has been showed that frequent rebalancing can eat up the potential excess return of such strategies \citep{bauer2004timing}. Consequently, their practical application is limited.